%%%%%%%%%%%%%%%%%%%%%%%%%%%%%%%%%%%%%%%%%
% Journal Article
% LaTeX Template
% Version 1.4 (15/5/16)
%
% This template has been downloaded from:
% http://www.LaTeXTemplates.com
%
% Original author:
% Frits Wenneker (http://www.howtotex.com) with extensive modifications by
% Vel (vel@LaTeXTemplates.com)
%
% License:
% CC BY-NC-SA 3.0 (http://creativecommons.org/licenses/by-nc-sa/3.0/)
%
%%%%%%%%%%%%%%%%%%%%%%%%%%%%%%%%%%%%%%%%%

%----------------------------------------------------------------------------------------
%	PACKAGES AND OTHER DOCUMENT CONFIGURATIONS
%----------------------------------------------------------------------------------------

\documentclass[twoside,twocolumn]{article}

\makeatletter
\renewcommand{\fnum@figure}{Fig. \thefigure}
\makeatother


\usepackage{blindtext} % Package to generate dummy text throughout this template 

\usepackage[sc]{mathpazo} % Use the Palatino font
\usepackage[T1]{fontenc} % Use 8-bit encoding that has 256 glyphs
\linespread{1.05} % Line spacing - Palatino needs more space between lines
\usepackage{microtype} % Slightly tweak font spacing for aesthetics

\usepackage[english]{babel} % Language hyphenation and typographical rules

\usepackage[hmarginratio=1:1,top=32mm,columnsep=20pt]{geometry} % Document margins
\usepackage[hang, small,labelfont=bf,up,textfont=it,up]{caption} % Custom captions under/above floats in tables or figures
\usepackage{booktabs} % Horizontal rules in tables
\usepackage{float}
\usepackage{lettrine} % The lettrine is the first enlarged letter at the beginning of the text

\usepackage{enumitem} % Customized lists
\setlist[itemize]{noitemsep} % Make itemize lists more compact
\usepackage{graphicx}

\usepackage{abstract} % Allows abstract customization
\renewcommand{\abstractnamefont}{\normalfont\bfseries} % Set the "Abstract" text to bold
\renewcommand{\abstracttextfont}{\normalfont\small\itshape} % Set the abstract itself to small italic text

\usepackage{titlesec} % Allows customization of titles
\renewcommand\thesection{\Roman{section}} % Roman numerals for the sections
\renewcommand\thesubsection{\roman{subsection}} % roman numerals for subsections
\titleformat{\section}[block]{\large\scshape\centering}{\thesection.}{1em}{} % Change the look of the section titles
\titleformat{\subsection}[block]{\large}{\thesubsection.}{1em}{} % Change the look of the section titles

\usepackage{fancyhdr} % Headers and footers
\pagestyle{fancy} % All pages have headers and footers
\fancyhead{} % Blank out the default header
\fancyfoot{} % Blank out the default footer
\fancyhead[C]{Running title $\bullet$ December 2017 $\bullet$ Aalborg University} % Custom header text
\fancyfoot[RO,LE]{\thepage} % Custom footer text

\usepackage{titling} % Customizing the title section

\usepackage{hyperref} % For hyperlinks in the PDF

\usepackage[utf8]{inputenc}
\usepackage[numbers]{natbib}	

%----------------------------------------------------------------------------------------
%	TITLE SECTION
%----------------------------------------------------------------------------------------

\setlength{\droptitle}{-4\baselineskip} % Move the title up

\pretitle{\begin{center}\Huge\bfseries} % Article title formatting
\posttitle{\end{center}} % Article title closing formatting
\title{\huge Prediction of symptom duration and pain intensity from patellofemoral pain maps using deep learning} % Article title
\author{%
\textsc{Birgithe Kleemann Rasmussen, Ignas Kupcikevičius,} \\
\textsc{Linette Helena Poulsen, Mads Kristensen}
 \\[1ex] % Your name
\normalsize Aalborg University \\ % Your institution
% Your email address
%\and % Uncomment if 2 authors are required, duplicate these 4 lines if more
%\textsc{Jane Smith}\thanks{Corresponding author} \\[1ex] % Second author's name
%\normalsize University of Utah \\ % Second author's institution
%\normalsize \href{mailto:jane@smith.com}{jane@smith.com} % Second author's email address
}
\date{December 20, 2017} % Leave empty to omit a date
\renewcommand{\maketitlehookd}{%

\begin{abstract}
\noindent
Introduction: Patellofemoral pain syndrome (PFPS) is a musculoskeletal condition that presents as pain behind or around the patella without known structural changes [1]. Partial correlations between perceived size of PFP from pain maps and symptom duration along with pain intensity has been indicated in previous studies [2], however morphology and location of PFP remains unexplored. Based on deep learnings objects detection capabilities, convolution methods can be used to detect image-features related to morphology. The aim of this study is to determine the performance of deep learning classification according to symptom duration and pain intensity, based on morphology and location of perceived PFP from pain maps. 

\noindent
Methods and materials: PFP drawings were collected on lower extremities body-schema and encoded into three different data representations in respect to morphology and location and a combination of the two. The distribution of the outputs were analyzed and used for defining the classification intervals for symptom duration (<12 and >36 months) and pain intensity (<4 and >8 on VAS). 
Estimation of generalization performance of the models was calculated through 5-fold cross validation during the training. Type 1 and type 2 errors were computed and results were compared between the three input representations.

\noindent
Results: The results showed that the combined representation performed with the highest accuracy (67\%). The location representation and morphology representation scored 63\% and 62\%, respectively, based on pain intensity. Generalization during training showed a higher accuracy for pain intensity classification (highest acc: 67\% SD: 6.78) than symptom duration (highest acc: 57.85\% SD: 4.95) using combined data representation.
(TEST RESULT MISSING: will come later)
 
 \noindent
Discussion: Despite pain intensity being defined as multidimensional and subjective, the performance accuracy were higher than that of symptom duration. The results may indicate that a combination of the morphology and the location of the pain influence the classification performance in relation to symptom duration or pain intensity. Currently, it is unclear if deep learning methods may be a suitable approach for classifying PFPS to work as support in a clinical setting, to which further investigation is necessary. Improvements could be found when more data become available to better reflect generalization patterns in PFP drawings.  

\end{abstract}
}

%----------------------------------------------------------------------------------------

\begin{document}

% Print the title
\maketitle

%----------------------------------------------------------------------------------------
%	ARTICLE CONTENTS
%----------------------------------------------------------------------------------------

\section{Introduction}
Patellofemoral pain syndrome (PFPS) is a painful musculoskeletal condition that is presented as pain behind or around patella \citep{Maclachlan2017, Smith2015}. PFPS affects 6-7 \% of adolescents, of whom two thirds are highly physically active \citep{Rathleff2015}. Additionally the prevalence is more than twice as high for females than males \citep{Petersen2013, Rathleff2015}.
PFPS may be present over longer periods of time where a high number of individuals experience a recurrent or chronic pain \citep{Witvrouw2014} and may also lead to osteoarthritis \citep{Petersen2013, Crossley2016}.

\noindent
Patellofemoral pain (PFP) is often described as diffuse knee pain, that can be hard to explain and localize \citep{Witvrouw2014}. Despite the fact that individuals feel pain in the knee, there is not any structural changes in the knee such as significant chondral damage or increased Q-angle. There is no definitive clinical test to diagnose PFPS and it is thereby often diagnosed based on exclusion criterias \citep{Petersen2013} to which PFPS is also described as an orthopaedic enigma, and is one of the most challenging pathologies to manage \citep{Dye2001}. 
To assist diagnosis of PFPS, pain maps may be used as a helpful tool for the individuals to communicate their pain by drawing pain areas \citep{Boudreau2016}. A study shows that through the use of pain maps it is possible to find a correlation between the symptom duration and the size and morphology of pain area \citep{Boudreau2017}. 
Another method to measure pain is by using visual analog scale (VAS), that scores pain between no-pain to the worst pain imaginable \citep{Haefeli2005}. However it is a known problem that chronic pain is considered a multidimensional pain, because the perceived pain of an individual is influenced by biomedical, psychosocial and behavioral factors \citep{Dansie2013}.

\noindent
Since PFP is associated with a lack of knowledge, and it has been shown that there is a correlation between pain maps and symptom duration as well as pain intensity, it is interesting to investigate if pain maps can be used to classify and thereby predict PFP related information. 

\noindent
A method that has not been found used in this context before is a deep learning. The deep learning method is chosen for this study because it is a state of the art method, that has shown greater performance in specific computation fields, compared to other machine learning methods \citep{LeCun2015}.
Furthermore, the method is chosen because of its ability to find non-linear connections between input and output data \citep{LeCun2015}, which is found relevant for this study mainly based on the fact that PFP is subjective and may be affected by the multidimensionality of chronic pain.  \\



\noindent
The goals of this project is to explore how accurate a deep learning model can classify symptom duration and pain intensity associated to PFP pain maps using a limited dataset. Because the prevalence is more than twice as high for females than males, the gender is included as a feature in the deep learning model. 
Furthermore, morphology of the pain is considered to be relevant, based on the indication that morphology and size of pain area increase with prolonged symptom duration. 
To investigate the influence of morphology and location of the pain three types of pain map representations are created: a binary representation which reflect the morphology, a simplified representation of morphology based on knee regions that give information about the pain location, and a combined representation that contains the morphology divided into knee regions.\\
\noindent
The aim of this study is to explore classification performance of a deep learning model, using PFP pain maps and gender as input to classify either symptom duration or pain intensity. 

\begin{center}
\textit{It is hypothesized that a deep learning model that uses pain maps and gender as input parameter has a higher performance when classifying according to symptom duration than pain intensity.}
\end{center}

\noindent
The secondary aim is to investigate if multiple pain map representations, which reflect the morphology and location of the pain, affect the deep learning model classification performance.                                                    
\begin{center}
\textit{It is hypothesized that different data represen-\newline tations of pain maps, reflecting morphology and location of pain, affect the performance
accuracy of a deep learning model when classifying according to symptom duration or pain intensity.
}
\end{center}



%------------------------------------------------

\section{Methods}
\textbf{Data and manual data handling} \newline
Data used in this study were collected beforehand from an on-going clinical trial (FOXH) which is conducted in collaboration with Danish and Australian universities. The data consists of pain maps which were drawn by individuals with PFPS through the use of an application, Navigate Pain, in a clinical setting. The pain maps are both from individuals with uni- and bilateral PFP, an example of these are shown in fig. \ref{fig:twoPainmaps}.

\begin{figure}[H]
\centering
\includegraphics[width=0.4\textwidth]{Figures/twoPainmaps}
\caption{Pain maps from individuals with uni- and bilateral PFP. The red markings indicate the area of pain perceived by the individuals.}
\label{fig:twoPainmaps}
\end{figure}

\noindent
In addition to the pain maps related information regarding the individuals was available.
Before using the data in the deep learning models, a manual data handling was necessary to match the given pain maps and associated ID on the individuals, which resulted in 217 available pain maps. Furthermore, specific information like gender, symptom duration and pain intensity were collected from the appurtenant information. The number of pain maps with associated gender and symptom duration, was 205. Additionally, there were 197 pain maps with associated gender and pain intensity.\\

\noindent
\textbf{Software application: Navigate Pain} \newline
Navigate Pain is a software application that is used to visualise the location, morphology and spatial distribution of pain from individuals to healthcare personnel. The application permits individuals to draw their pain with different colors and line thickness onto a body outline. Navigate Pain android was developed at Aalborg University and a commercial web application is available at Aglance Solutions (Denmark).\citep{Solutions2015}\\

\noindent
\textbf{Knee regions} \newline
\noindent
To define the location of the PFP the knees are divided into 20 regions, which are inspired by Photographic Knee Pain Map (PKPM). The divisions are designed to categorise location of knee pain for diagnostic and research purposes. PKPM represent both knees that makes it possible to identify unilateral and bilateral pain.\citep{Elson2010} The knee regions are illustrated in fig. \ref{fig:atlas}.

\begin{figure} [H] 
\centering
\includegraphics[width=0.38\textwidth]{Figures/atlas}
\caption{The regions of the left (L1-L10) and right (R1-R10) knees, where each knee is split into ten regions.}
\label{fig:atlas}
\end{figure}

\noindent
The regions are based on the anatomical structures according to the areas where individuals often indicate pain.
There are ten regions on each knee, where region 1 and 3 represent the superior lateral and superior medial areas for patella. Region 2 refers to quadriceps tendon. The patella is divided into lateral and medial regions, which are region 4 and 5. Region 6 and 8 are lateral and medial joint line areas. Patella tendor is region 7 and the two last regions, 9 and 10, are tibia lateral and medial.\citep{Elson2010}

\noindent
\textbf{Data representations} \newline
\noindent
To investigate whether morphology and location of pain have an influence on the outputs, symptom duration and pain intensity, the pain maps are encoded in multiple data representations. The pain maps were processed in MatLab, where the images were resized, since they were collected at different resolutions (screen sizes) and cropped to sort out unnecessary data like the areas inferior and superior to the knee.  Each data representation is reflected in a matrix consisting of the pain maps, gender and the output, symptom duration and pain intensity. Since the original pain maps reflecting the morphology of the pain, thus the morphology-representation does not require further manipulation. \newline
\noindent
To investigate whether the location alone have a correlation to the outputs, a simplified representation of the pain maps are created. The location of the pain is then reflected by the use of the defined knee regions (fig. \ref{fig:atlas}), where each region represent a value of 0 (not active) or 1 (active) in a vector.  The values were defined by using a threshold to determine whether a region was considered active in relation the amount of pain. A threshold was required to increase the confidence of an active pain region by avoiding minimal contributions e.g. small pain areas in the associated regions. Simultaneously the threshold should not be too large so that pain areas was excluded. The threshold was decided based on an analysis on five random pain maps, where threshold values of 0, 5, 10 and 15\% was compared. The threshold represent which minimal percentage of pain should be present in a specific region before it is considered active. Based on the analysis a 5\% threshold was chosen. \newline
\noindent
Lastly, a data representation which reflects a combination of morphology and location of the pain, is prepared to explore if the interaction of morphology and location of pain would give a better classification according to the outputs.


- Simple regression
- Deep learning + modeller


%------------------------------------------------

\section{Results}

\begin{figure*} [b!]
\begin{tcolorbox}[colframe=black!30!black, colback=white]
\hfill
\begin{subfigure}[r]{0.5\textwidth}
    \includegraphics[width=\textwidth]{Figures/legend}
  \end{subfigure}
  \vskip\baselineskip
  \hspace{-5mm}
  \begin{subfigure}[b]{0.51\textwidth}
    \includegraphics[width=\textwidth]{Figures/durapixel}
    \caption{ }
    \label{fig:1}
  \end{subfigure}
  \hfill
    \hspace{2mm}
  \begin{subfigure}[b]{0.51\textwidth}
    \includegraphics[width=\textwidth]{Figures/duraregion}
       \caption{ }
    \label{fig:2}
  \end{subfigure}
    \vskip\baselineskip
    \hspace{-5mm}
  \begin{subfigure}[b]{0.51\textwidth}
    \includegraphics[width=\textwidth]{Figures/vaspixel}
    \caption{}
    \label{fig:3}
  \end{subfigure}
  \hfill
  \hspace{2mm}
  \begin{subfigure}[b]{0.51\textwidth}
    \includegraphics[width=\textwidth]{Figures/vasregion}
       \caption{ }
    \label{fig:4}
  \end{subfigure}  
  \caption{Linear correlations of pain pixels and pain duration (a), active pain regions and pain duration (b), pain pixels and pain intensity indicated in VAS (c), and active pain regions and pain intensity indicated in VAS (d).}
  \label{fig:correlations}
\end{tcolorbox}
\end{figure*}

This section visualizes the results from the linear regressions, and performance of the deep learning models using multiple pain map representations, and different outputs. 
\vspace{-0.3cm}

\subsection{Linear correlations}
The linear regression between simple features, number of pain pixels or active pain regions, and outputs, pain duration or pain intensity, resulted in the plots shown in fig. \ref{fig:correlations}. The $R^2$-values support the nonlinearity, shown in the plots, where correlation fig. \ref{fig:1} resulted in a $R^2 = 0.018$, fig. \ref{fig:2} resulted in $R^2 = 0.008$, fig. \ref{fig:3} resulted in $R^2 = 0.011$ and fig. \ref{fig:4} resulted in $R^2 = 0.011$. 


\begin{table*}[b!]
\centering
\begin{tabular}{@{}llll@{}}
\toprule
\multicolumn{4}{c}{\hspace{2.3cm} Avg. accuracy (\%) \hspace{1cm} Avg. sensitivity (\%) \hspace{1cm} Avg. specificity (\%) \hspace{1cm}                }                                                                                                                                                                         \\ \midrule
\multicolumn{4}{c}{Morphology-representation} \\ \midrule
Pain duration  & \hspace{0.7cm}\begin{tabular}[c]{@{}l@{}}69.44\% \end{tabular} & \hspace{2.6cm} \begin{tabular}[c]{@{}l@{}}56.25\%\end{tabular} & \hspace{2.7cm} \begin{tabular}[c]{@{}l@{}} 80.00\%\end{tabular} \\ %\midrule
Pain intensity & \hspace{0.6cm} \begin{tabular}[c]{@{}l@{}}60.00\% \end{tabular}  & \hspace{2.7cm}\begin{tabular}[c]{@{}l@{}}40.00\% \end{tabular}  & \hspace{2.7cm} \begin{tabular}[c]{@{}l@{}}75.00\%\end{tabular}   \\ \midrule
\multicolumn{4}{c}{Location-representation}                                                                                                                                                                             \\ \midrule
Pain duration  & \hspace{0.6cm} \begin{tabular}[c]{@{}l@{}}35.29\%\end{tabular}  & \hspace{2.6cm} \begin{tabular}[c]{@{}l@{}}0.00\% \end{tabular}  & \hspace{2.7cm} \begin{tabular}[c]{@{}l@{}}100\%\end{tabular}  \\ %\midrule 
Pain intensity & \hspace{0.6cm} \begin{tabular}[c]{@{}l@{}}60.71\% \end{tabular}   & \hspace{2.6cm} \begin{tabular}[c]{@{}l@{}}0.00\% \end{tabular}   & \hspace{2.7cm} \begin{tabular}[c]{@{}l@{}}100\%\end{tabular}  \\ \midrule
\multicolumn{4}{c}{Combined-representation}                                                                                                                                                              \\ \midrule
Pain duration  & \hspace{0.6cm} \begin{tabular}[c]{@{}l@{}}55.56\% \end{tabular}                                                                   & \hspace{2.6cm} \begin{tabular}[c]{@{}l@{}}55.00\%\end{tabular}                                                                & \hspace{2.7cm} \begin{tabular}[c]{@{}l@{}}43.75\%\end{tabular}                                                                                                                                \\% \midrule
Pain intensity & \hspace{0.6cm} \begin{tabular}[c]{@{}l@{}}86.67\% \end{tabular}
&\hspace{2.6cm} \begin{tabular}[c]{@{}l@{}} 75.00\%\end{tabular}                                                               
& \hspace{2.7cm} \begin{tabular}[c]{@{}l@{}} 90.91\%\end{tabular}                                                               
 \\ \bottomrule
\end{tabular}
\caption{Generalization performance of the models, which use the MR, LR, and CR when classifying according to pain duration or pain intensity.}
\label{tab:performance}
\end{table*}


\begin{figure*} [b!]
\begin{tcolorbox}[colframe=black!30!black, colback=white]
    \includegraphics[width=1\textwidth]{Figures/samcon}
  \caption{Confusion matrices of (a) MR classified according to pain duration, (b) classified LR according pain duration, and (c) classified CR according to pain duration.}
  \label{fig:confma}
\end{tcolorbox}
\end{figure*}

\begin{figure*} [t!]
\begin{tcolorbox}[colframe=black!30!black, colback=white]
    \includegraphics[width=1\textwidth]{Figures/samcon1}
  \caption{Confusion matrices of (a) MR classified according to pain intensity, (b) classified LR according pain intensity, and (c) classified CR according to pain intensity.}
  \label{fig:confma1}
\end{tcolorbox}
\end{figure*}

\subsection{Optimization of the models}
During the optimization, a structured grid search resulted in different hyperparameters according to each model. For the deep learning models including MR and classification type pain duration, the highest performance was obtained with a learning rate of 0.01, the number of nodes of 64, and the epochs and batch size setting of 120 and 20. The models using the MR and pain intensity had the highest performance when using a learning rate of 0.1, 16 nodes, epochs and batch size of 140 and 10. The models including the LR had similar results from the optimization. Learning rate of 0.01, number of nodes at 16, and a number of epochs and batch size of 120 and 20. 
Results of optimization on the models including the CR were almost identical. Both resulted in the best performance with 16 nodes, and with a number of epochs and batch size of 120 and 30, to which the only difference was in the learning rate that for pain duration was 0.1, and 0.001 for pain intensity.

\subsection{Performance of the models}
The average performance accuracy, sensitivity, and specificity of the models during test with new pain maps in different representations are shown in tab. \ref{tab:performance}. \newline
For the pain map representations, confusion matrices were created according to the pain duration as is shown in fig. \ref{fig:confma}, and confusion matrices according to pain intensity as shown in fig. \ref{fig:confma1}.

%------------------------------------------------

\section{Discussion}
This section discusses complexity of pain maps, and what may optimize the performance of the models. Furthermore, the results are discussed, whereas the highest performance value of the pain maps representations is evaluated. Finally, the performance according to the output, pain duration or pain intensity, is discussed.

\subsection*{Amount of pain maps}
In this study the total number of pain maps was 217 from individual with uni- and bilateral PFP. Comparing to the literature, a supervised deep learning model should use five thousand labeled data per category to obtain an acceptable performance \citep{Goodfellow2016}. 
The amount of pain maps was not optimal, to which a split body approach was used to compensate this. During this approach combined with the mirroring pain to the right knee, it was assumed that pain duration and pain intensity were identical for both knees. Theoretically, the bilateral PFP may have occurred on one leg first, and afterwards have spreaded to the other knee, which could affect the pain duration. Furthermore, individuals with bilateral pain may feel more pain on one of the knees. This may have resulted in incorrect labeled pain maps, which could influence the performance accuracy of the models.

\subsection*{Classification of pain maps}
The $R^2$-values of the linear correlations were close to zero, which represents nonlinearity, meaning that there are no simple correlation between the simple features and the outputs, pain duration and pain intensity. Thus, a deep learning model is used to investigate the complexity of pain maps.
The deep learning models could classify both pain duration and pain intensity classes in the MR. Models including LR classified only according to the higher intervals (pain duration above 36 months and pain intensity above 8 VAS), and appears unable to classify according to the lower classes (pain duration below 12 month and pain intensity below 4 VAS). This can be observed in the confusion matrices of LR, where sensitivity of both inputs were equal to 0\%. CR, classified according to the pain duration, found the patterns in both classes, but overall accuracy was 55.56\%, which could be caused by including the knee regions as one of the features.
Based on the results, the accuracy of MR, scored higher compared to LR. It can be discussed, that the morphology can be considered to be a better classification feature. Models containing MR scored 69.44\% and 60\% while LR scored 35.29\% and 60.71\%, reflecting pain duration and pain intensity.
Low score in LR could be considered as a result of imbalance between the duration intervals, as it could be also influenced with a further optimization. 
The low score could also be the result of the simplification of the location of the pain, that leads to the models not being able to find a pattern between the input and target output.  

\subsection*{Threshold} 
The LR had a 5\% threshold that defined when a pain region was considered active according to the amount of pain. It can be discussed whether this threshold was suitable, since adding the threshold resulted in loss of pain maps that had a very small amount of pain. However, a smaller threshold or no threshold would give active pain regions that might only contain very few pain pixels. Since PFP is described as hard to localize, it is unknown how precise the individuals have drawn their pain, thus every pixel should maybe not be taken into account. 
The CR did not have a threshold for defining active pain regions, because the morphology of the pain would be affected when discarding small pain regions. This representation is not a complete combination of the MR and LR.

\subsection*{Classification according to output}
Generalization performance of the six models showed that pain duration was a more stable classifier compared to pain intensity. As a result, both classes of pain duration can be predicted in the MR and CR, while classes representing pain intensity can be classified only with the morphology-representation.
It could be discussed that the lower results of pain intensity against the pain duration, might be caused by that the pain intensity is a subjective statement and is considered as multifactorial, while pain duration is an objective parameter, and it is thereby expected that the models have a higher performance when classifying pain maps according to pain duration. 
Given the fact that the sensitivity for multiple models was 0\%, the accuracy becomes a reflection of the imbalance between classes in the test set, e.g. if the test subset contained 75\% of the class with high pain intensity, the accuracy would also be 75\%. This may also be a result of the limited amount of pain maps used in this study.  
It is unknown whether the reason for the sensitivity to be 0\% is caused by that the models cannot find patterns according to the low valued pain intensity (0 to 4 VAS), or it could be a result of the models being poor, and classifies all inputs as high pain intensity (8 to 10 VAS).  

\subsection*{Optimization of deep learning models}
Optimization of the deep learning models is often a time-consuming process based on the picks of the multiple hyperparameters and different algorithms which could be implemented during the development of the models.  
Activation functions were chosen based on the literature, where ReLU should be used for convolutional and fully connected layers in neural network models, and sigmoid should be picked for binary output layer. However, additional testing could be made by using softmax or linear activation function to increase the generalization performance. The dropout algorithm was set to the default 0.5 and used in all models between the fully connected layers to turn off the amount of nodes and prevent the models from overfitting. Additional values could have been tested in order to find most optimal for the every model.
Unfortunately, the lack of time and time-consuming reruns during every optimization cycle, lead to use the common hyperparameters as there were many other which were tested with grid search 10-fold cross validation.
A limitation for this study was the available computational power for training of the model to which an improvement in performance may be found through more powerful systems or services. \\
\noindent
A further optimization of the models may be found according to the input parameters, to which more physical, and psychological features may increase the performance accuracy. Physical features, such as age, height, weight, physical activity level, and sport activity, may influence pain duration or pain intensity. Age could be a relevant feature since the perceived pain is dependent on the individual's personality and character. Younger individuals may feel more pain because of a new pain, than older individuals which have had PFP for a longer period of time. In addition, older individuals may feel more pain because of the phenomenon central sensitization, which in some cases results in widespread pain. The physical activity level, and sport may increase the pain intensity for some individuals because of the patellofemoral loaded activity. Psychological factor is an important feature to consider, because of its influence on pain intensity. Pain is multifactorial and can be influenced of psychosocial factors \citep{Roos2003}. Furthermore, other pain areas, such as hip pain, may influence the pain intensity of PFP.
It may be considered to include either pain duration or pain intensity as an input to classify according to either pain intensity or pain duration, because of the possibility that there is a correlation between the two.

%------------------------------------------------

\section{Conclusion}
This study predicted pain maps according to the pain duration or pain intensity.
The CR performed with the highest accuracy according to pain intensity, despite the subjectivity of pain and the influence of multidimensional factors.
There may be an indication of patterns to be found between pain maps and pain duration or pain intensity, however further optimization, more pain maps or additional studies are needed to support whether location and morphology contain positive predictive value in terms of predicting pain duration or pain intensity.





%----------------------------------------------------------------------------------------
%	REFERENCE LIST
%----------------------------------------------------------------------------------------

%\begin{thebibliography}{99} % Bibliography - this is intentionally simple in this template
%
%\bibitem[Figueredo and Wolf, 2009]{Figueredo:2009dg}
%Figueredo, A.~J. and Wolf, P. S.~A. (2009).
%\newblock Assortative pairing and life history strategy - a cross-cultural
%  study.
%\newblock {\em Human Nature}, 20:317--330.
% 
%\end{thebibliography}

%----------------------------------------------------------------------------------------
\begingroup
\raggedright

\bibliographystyle{unsrtnat}
\bibliography{bib}
%\urlstyle{same}
%
%
%\printbibliography
%\cleardoublepage

\endgroup


\end{document}
