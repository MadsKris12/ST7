This section discusses complexity of pain maps, and what may optimize the performance of the models. Furthermore, the results are discussed, whereas the highest performance value of the pain maps representations is evaluated. Finally, the performance according to the output, pain duration or pain intensity, is discussed.

\subsection*{Amount of pain maps}
In this study the total number of pain maps was 217 from individual with uni- and bilateral PFP. Comparing to the literature, a supervised deep learning model should use five thousand labeled data per category to obtain an acceptable performance \citep{Goodfellow2016}. 
The amount of pain maps was not optimal, to which a split body approach was used to compensate this. During this approach combined with the mirroring pain to the right knee, it was assumed that pain duration and pain intensity were identical for both knees. Theoretically, the bilateral PFP may have occurred on one leg first, and afterwards have spreaded to the other knee, which could affect the pain duration. Furthermore, individuals with bilateral pain may feel more pain on one of the knees. This may have resulted in incorrect labeled pain maps, which could influence the performance accuracy of the models.

\subsection*{Classification of pain maps}
The $R^2$-values of the linear correlations were close to zero, which represents nonlinearity, meaning that there are no simple correlation between the simple features and the outputs, pain duration and pain intensity. Thus, a deep learning model is used to investigate the complexity of pain maps.
The deep learning models could classify both pain duration and pain intensity classes in the MR. Models including LR classified only according to the higher intervals (pain duration above 36 months and pain intensity above 8 VAS), and appears unable to classify according to the lower classes (pain duration below 12 month and pain intensity below 4 VAS). This can be observed in the confusion matrices of LR, where sensitivity of both inputs were equal to 0\%. CR, classified according to the pain duration, found the patterns in both classes, but overall accuracy was 55.56\%, which could be caused by including the knee regions as one of the features.
Based on the results, the accuracy of MR, scored higher compared to LR. It can be discussed, that the morphology can be considered to be a better classification feature. Models containing MR scored 69.44\% and 60\% while LR scored 35.29\% and 60.71\%, reflecting pain duration and pain intensity.
Low score in LR could be considered as a result of imbalance between the duration intervals, as it could be also influenced with a further optimization. 
The low score could also be the result of the simplification of the location of the pain, that leads to the models not being able to find a pattern between the input and target output.  

\subsection*{Threshold} 
The LR had a 5\% threshold that defined when a pain region was considered active according to the amount of pain. It can be discussed whether this threshold was suitable, since adding the threshold resulted in loss of pain maps that had a very small amount of pain. However, a smaller threshold or no threshold would give active pain regions that might only contain very few pain pixels. Since PFP is described as hard to localize, it is unknown how precise the individuals have drawn their pain, thus every pixel should maybe not be taken into account. 
The CR did not have a threshold for defining active pain regions, because the morphology of the pain would be affected when discarding small pain regions. This representation is not a complete combination of the MR and LR.

\subsection*{Classification according to output}
Generalization performance of the six models showed that pain duration was a more stable classifier compared to pain intensity. As a result, both classes of pain duration can be predicted in the MR and CR, while classes representing pain intensity can be classified only with the morphology-representation.
It could be discussed that the lower results of pain intensity against the pain duration, might be caused by that the pain intensity is a subjective statement and is considered as multifactorial, while pain duration is an objective parameter, and it is thereby expected that the models have a higher performance when classifying pain maps according to pain duration. 
Given the fact that the sensitivity for multiple models was 0\%, the accuracy becomes a reflection of the imbalance between classes in the test set, e.g. if the test subset contained 75\% of the class with high pain intensity, the accuracy would also be 75\%. This may also be a result of the limited amount of pain maps used in this study.  
It is unknown whether the reason for the sensitivity to be 0\% is caused by that the models cannot find patterns according to the low valued pain intensity (0 to 4 VAS), or it could be a result of the models being poor, and classifies all inputs as high pain intensity (8 to 10 VAS).  

\subsection*{Optimization of deep learning models}
Optimization of the deep learning models is often a time-consuming process based on the picks of the multiple hyperparameters and different algorithms which could be implemented during the development of the models.  
Activation functions were chosen based on the literature, where ReLU should be used for convolutional and fully connected layers in neural network models, and sigmoid should be picked for binary output layer. However, additional testing could be made by using softmax or linear activation function to increase the generalization performance. The dropout algorithm was set to the default 0.5 and used in all models between the fully connected layers to turn off the amount of nodes and prevent the models from overfitting. Additional values could have been tested in order to find most optimal for the every model.
Unfortunately, the lack of time and time-consuming reruns during every optimization cycle, lead to use the common hyperparameters as there were many other which were tested with grid search 10-fold cross validation.
A limitation for this study was the available computational power for training of the model to which an improvement in performance may be found through more powerful systems or services. \\
\noindent
A further optimization of the models may be found according to the input parameters, to which more physical, and psychological features may increase the performance accuracy. Physical features, such as age, height, weight, physical activity level, and sport activity, may influence pain duration or pain intensity. Age could be a relevant feature since the perceived pain is dependent on the individual's personality and character. Younger individuals may feel more pain because of a new pain, than older individuals which have had PFP for a longer period of time. In addition, older individuals may feel more pain because of the phenomenon central sensitization, which in some cases results in widespread pain. The physical activity level, and sport may increase the pain intensity for some individuals because of the patellofemoral loaded activity. Psychological factor is an important feature to consider, because of its influence on pain intensity. Pain is multifactorial and can be influenced of psychosocial factors \citep{Roos2003}. Furthermore, other pain areas, such as hip pain, may influence the pain intensity of PFP.
It may be considered to include either pain duration or pain intensity as an input to classify according to either pain intensity or pain duration, because of the possibility that there is a correlation between the two.