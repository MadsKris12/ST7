This section discusses complexity of pain maps, and what may optimize the performance of the models. Furthermore, the results are discussed, whereas the highest performance value of the pain map representations is evaluated. Finally, the performance according to the output, pain duration or pain intensity, is discussed.

\subsection{Amount of pain maps}
In this study the total number of pain maps was 217 from individual with uni- and bilateral PFP. Comparing to the literature, a supervised deep learning model should use five thousand labeled data per category to obtain an acceptable performance \citep{Goodfellow2016}.
The amount of pain maps was not optimal, to which a split body approach was used to compensate this. During this approach combined with the mirroring pain to the right knee, it was assumed that pain duration and pain intensity were identical for both knees. Theoretically, the bilateral PFP may have occurred on one leg first, and afterwards have spreaded to the other knee, which could affect the pain duration. Furthermore, individuals with bilateral pain may feel more pain on one of the knees. This may have resulted in incorrect labeled pain maps, which could influence the performance accuracy of the models.

\subsection{Classification of pain maps}
The $R^2$-values of the linear correlations were close to zero, which represents nonlinearity. Thus, a deep learning model used to investigate the complexity of pain maps.
The models that used the MR resulted in a higher predictive value for pain duration and pain intensity in the higher extremes (36+ months and 8-10 VAS), compared to the lower extremes (0-12 months and 0-4 VAS). By comparing the sensitivity and specificity of the model using MR to predict pain duration or pain intensity, the model predicting the pain duration was better at predicting according to the true low pain duration (0-12 months) and true high pain duration (36-300 months). The models representing LR could not distinguish between the pain maps according to the extremes for both pain duration and pain intensity. The models simply classified all pain maps as being in the higher classes. 
The model using CR classified better according to pain duration when predicting pain maps with a short duration, where the model that classified according to pain intensity was better to predict the higher pain intensities. 
Overall the accuracy indicates whether a pattern may be present in the pain maps according to the given classification, however this is not the case when the sensitivity or specificity is zero valued, to which the accuracy only reflects the amount data there is for one class. Bases on the accuracy, sensitivity and specificity for the models, the model using the CR classified according to pain intensity had the highest predictive value.

\subsection{Threshold}
The LR had a 5\% threshold that defined when a pain region was considered active according to the amount of pain. It can be discussed whether this threshold was suitable, since adding the threshold resulted in loss of pain maps that had a very small amount of pain. However, a smaller threshold or no threshold would give active pain regions that might only contain very few pain pixels. Since PFP is described as hard to localize, it is unknown how precise the individuals have drawn their pain, thus every pixel should maybe not be taken into account.
The CR did not have a threshold for defining active pain regions, because the morphology of the pain would be affected when discarding small pain regions. This representation is not a complete combination of the MR and LR.

\subsection{Optimization of deep learning models}
Optimization of the deep learning models is often a time-consuming process based on the picks of the multiple hyperparameters and different algorithms which could be implemented during the development of the models. 
Activation functions were chosen based on the literature, where ReLU should be used for convolutional and fully connected layers in neural network models, and sigmoid should be picked for binary output layer. However, additional testing could be made by using softmax or linear activation function to increase the generalization performance. The dropout algorithm was set to the default 0.5 and used in all models between the fully connected layers to turn off the amount of nodes and prevent the models from overfitting. Additional values could have been tested in order to find most optimal for the every model.
Unfortunately, the lack of time and time-consuming reruns during every optimization cycle, lead to use the common hyperparameters as there were many other which were tested with grid search 10-fold cross validation.
A limitation for this study was the available computational power for training of the model to which an improvement in performance may be found through more powerful systems or services. \\
\noindent
A further optimization of the models may be found according to the input parameters, to which more physical, and psychological features may increase the performance accuracy. Physical features, such as age, height, weight, physical activity level, and sport activity, may influence pain duration or pain intensity. Age could be a relevant feature since the perceived pain is dependent on the individual's personality and character. Younger individuals may feel more pain because of a new pain, than older individuals which have had PFP for a longer period of time. In addition, older individuals may feel more pain because of the phenomenon central sensitization, which in some cases results in widespread pain. The physical activity level, and sport may increase the pain intensity for some individuals because of the patellofemoral loaded activity. Psychological factor is an important feature to consider, because of its influence on pain intensity. Pain is multifactorial and can be influenced of psychosocial factors \citep{Roos2003}. Furthermore, other pain areas, such as hip pain, may influence the pain intensity of PFP.
