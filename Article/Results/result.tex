
\begin{figure*} [b!]
\begin{tcolorbox}[colframe=black!30!black, colback=white]
\hfill
\begin{subfigure}[r]{0.5\textwidth}
    \includegraphics[width=\textwidth]{Figures/legend}
  \end{subfigure}
  \vskip\baselineskip
  \hspace{-5mm}
  \begin{subfigure}[b]{0.51\textwidth}
    \includegraphics[width=\textwidth]{Figures/durapixel}
    \caption{ }
    \label{fig:1}
  \end{subfigure}
  \hfill
    \hspace{2mm}
  \begin{subfigure}[b]{0.51\textwidth}
    \includegraphics[width=\textwidth]{Figures/duraregion}
       \caption{ }
    \label{fig:2}
  \end{subfigure}
    \vskip\baselineskip
    \hspace{-5mm}
  \begin{subfigure}[b]{0.51\textwidth}
    \includegraphics[width=\textwidth]{Figures/vaspixel}
    \caption{}
    \label{fig:3}
  \end{subfigure}
  \hfill
  \hspace{2mm}
  \begin{subfigure}[b]{0.51\textwidth}
    \includegraphics[width=\textwidth]{Figures/vasregion}
       \caption{ }
    \label{fig:4}
  \end{subfigure}  
  \caption{Linear correlations of pain pixels and pain duration (a), active pain regions and pain duration (b), pain pixels and pain intensity indicated in VAS (c), and active pain regions and pain intensity indicated in VAS (d).}
  \label{fig:correlations}
\end{tcolorbox}
\end{figure*}

This section visualizes the results from the linear regressions, and performance of the deep learning models using multiple pain map representations, and different outputs. 
\vspace{-0.3cm}

\subsection*{Linear correlations}
The linear regression between simple features, number of pain pixels or active pain regions, and outputs, pain duration or pain intensity, resulted in the plots shown in fig. \ref{fig:correlations}. The $R^2$-values support the nonlinearity, shown in the plots, where correlation fig. \ref{fig:1} resulted in a $R^2 = 0.018$, fig. \ref{fig:2} resulted in $R^2 = 0.008$, fig. \ref{fig:3} resulted in $R^2 = 0.011$ and fig. \ref{fig:4} resulted in $R^2 = 0.011$. 


\begin{table*}[b!]
\centering
\begin{tabular}{@{}llll@{}}
\toprule
\multicolumn{4}{c}{\hspace{2.3cm} Avg. accuracy (\%) \hspace{1cm} Avg. sensitivity (\%) \hspace{1cm} Avg. specificity (\%) \hspace{1cm}                }                                                                                                                                                                         \\ \midrule
\multicolumn{4}{c}{Morphology-representation} \\ \midrule
Pain duration  & \hspace{0.7cm}\begin{tabular}[c]{@{}l@{}}69.44\% \end{tabular} & \hspace{2.6cm} \begin{tabular}[c]{@{}l@{}}69.23\%\end{tabular} & \hspace{2.7cm} \begin{tabular}[c]{@{}l@{}} 69.57\%\end{tabular} \\ %\midrule
Pain intensity & \hspace{0.6cm} \begin{tabular}[c]{@{}l@{}}60.00\% \end{tabular}  & \hspace{2.7cm}\begin{tabular}[c]{@{}l@{}}40.00\% \end{tabular}  & \hspace{2.7cm} \begin{tabular}[c]{@{}l@{}}70.00\%\end{tabular}   \\ \midrule
\multicolumn{4}{c}{Location-representation}                                                                                                                                                                             \\ \midrule
Pain duration  & \hspace{0.6cm} \begin{tabular}[c]{@{}l@{}}35.29\%\end{tabular}  & \hspace{2.6cm} \begin{tabular}[c]{@{}l@{}}0.00\% \end{tabular}  & \hspace{2.7cm} \begin{tabular}[c]{@{}l@{}}35.29\%\end{tabular}  \\ %\midrule 
Pain intensity & \hspace{0.6cm} \begin{tabular}[c]{@{}l@{}}60.71\% \end{tabular}   & \hspace{2.6cm} \begin{tabular}[c]{@{}l@{}}0.00\% \end{tabular}   & \hspace{2.7cm} \begin{tabular}[c]{@{}l@{}}60.71\%\end{tabular}  \\ \midrule
\multicolumn{4}{c}{Combined-representation}                                                                                                                                                              \\ \midrule
Pain duration  & \hspace{0.6cm} \begin{tabular}[c]{@{}l@{}}55.56\% \end{tabular}                                                                   & \hspace{2.6cm} \begin{tabular}[c]{@{}l@{}}61.11\%\end{tabular}                                                                & \hspace{2.7cm} \begin{tabular}[c]{@{}l@{}}50.00\%\end{tabular}                                                                                                                                \\% \midrule
Pain intensity & \hspace{0.6cm} \begin{tabular}[c]{@{}l@{}}73.33\% \end{tabular}
&\hspace{2.6cm} \begin{tabular}[c]{@{}l@{}} 0.00\%\end{tabular}                                                               
& \hspace{2.7cm} \begin{tabular}[c]{@{}l@{}} 73.33\%\end{tabular}                                                               
 \\ \bottomrule
\end{tabular}
\caption{Generalization performance of the models, which use the MR, LR, and CR when classifying according to pain duration or pain intensity.}
\label{tab:performance}
\end{table*}


\begin{figure*} [b!]
\begin{tcolorbox}[colframe=black!30!black, colback=white]
    \includegraphics[width=1\textwidth]{Figures/samcon}
  \caption{Confusion matrices of (a) MR classified according pain duration, (b) LR classified according to pain intensity, and (c) CR classified according to pain intensity.}
  \label{fig:confma}
\end{tcolorbox}
\end{figure*}

\subsection*{Optimization of the models}
During the optimization, a structured grid search resulted in different hyperparameters according to each model. The learning rate was different for the models including MR according to pain duration (0.02), and pain intensity (0.1). An optimization in kernel initializer was defined as glorot\_uniform for pain duration, and glorot\_normal for pain intensity. The nodes were changed from 32 in the fully connected layers to 64 for pain duration, and 16 for pain intensity. Lastly, an optimization was made for number of epochs, and batch size, which were changed to 120 and 20 for pain duration, and 140 and 10 for pain intensity. The models including the LR had similar results from the optimization. A learning rate of 0.01, a glorot\_uniform kernel initializer, 16 nodes, and number of epochs and batch size of 120 and 20. 
Results of optimization on the models including the CR were almost identical. Both resulted in the best performance with a glorot\_uniform kernel initializer, 16 nodes, and with a number of epochs and batch size of 120 and 30, to which the only difference was in the learning rate that for pain duration was 0.1, and 0.001 for pain intensity.

\subsection*{Performance of the models}
The average performance accuracy, sensitivity, and specificity of the models during test with new pain maps in different representations are shown in tab. \ref{tab:performance}. \newline
For the pain map representations that resulted in the highest accuracy for either pain duration or pain intensity, confusion matrices were created, and is shown in fig. \ref{fig:confma}.