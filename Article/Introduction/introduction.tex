Patellofemoral pain syndrome (PFPS) is a painful musculoskeletal condition[3, 5], which presents pain behind or around the patella. The patellofemoral pain is often known as anterior knee pain and runner’s knee. The pain is often described like diffuse knee pain, which is provoked by patellofemoral loaded activities like climbing stairs, running on hard or slanted surfaces, hiking, squatting or just prolonged sitting in the same position. [MARTINI, 2, 4, 5, 6] The pain is therefore not caused by previous trauma[2].
Knee pain is not the only symptom of PFPS, the patient often complain about knee stiffness, patellofemoral crepitus, swelling knee and having trouble with common daily activities[MARTINI, 2]. The patients may limit or stop the physical activity because of the pain, and that can lead to weight gain[1, 2].

Physiologically PFP is associated with incorrect movement of the patellar, that occurs when the patella moves outside of its ordinary track, which for instance can be movement in lateral direction instead of movement superior-inferior direction.\citep{Martini2012}

The patellofemoral pain (PFP) is mostly prevalent in adolescents and younger adults who are physically active, but it can affect people of all ages and activity levels[2, 3, 4]. Additionally,  females are affected about more than twice as often as males[1]. Furthermore the PFPS can persist for up to 20 years and lead to osteoarthritis[1, 2].

Despite the patient feels pain in the knee, there isn’t any structural changes in the knee such as increased Q-angle or significant chondral damage [1]. There is no definitive clinical test to diagnose PFP, but there is a test to elicit the knee pain by during a squatting manoeuvre. The PFP is evident in 80 % of people who are tested positive in this test. [2, 4] Therefore the diagnosis PFPS is often based on exclusion[1]. After the diagnosis, can evidence based treatments reduce pain and improve function that allows patients to maintain physical activity[4].
The aetiology of PFPS still remains unclear[5].
