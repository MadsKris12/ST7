Patellofemoral pain (PFP) syndrome is a painful musculoskeletal condition that is presented as pain behind or around the patella \citep{Maclachlan2017, Smith2015}. PFP syndrome affects 6-7\% of adolescents, of whom two thirds are highly physically active \citep{Rathleff2015}. Additionally, the prevalence is more than twice as high for females than males \citep{Rathleff2015, Petersen2013}.
PFP syndrome is typically present over a longer period of time where a high number of individuals experience a recurrent or chronic pain \citep{Witvrouw2014}. Chronic pain may be maintained by the phenomenon central sensitization, which may resulted in widespread pain over time. Ultimately, PFP syndrome may lead to osteoarthritis \citep{Petersen2013, Crossley2016}. 

\noindent
PFP is often described as diffuse knee pain, that can be hard for individuals to explain and localize \citep{Witvrouw2014}. Despite that individuals feel pain in the knee, there is no underlying structural changes in the knee such as significant chondral damage. There is no definitive clinical test to diagnose PFP syndrome and is often diagnosed based on exclusion criterias \citep{Petersen2013}, to which PFP syndrome is also described as an orthopaedic enigma, and is one of the most challenging pathologies to manage \citep{Dye2001}.
To assist diagnosis of PFP syndrome, pain maps may be used as a helpful tool for the individuals to communicate their pain by drawing pain areas on a body outline \citep{Boudreau2016}.

\noindent
A study by \citeauthor{Boudreau2017} \citep{Boudreau2017} indicates, through the use of pain maps (n=35), that there is a correlation between the size of the pain areas (total number of pain pixels) and the pain duration as well as pain intensity for individuals with PFP duration longer than five years.\citep{Boudreau2017}
However, it is unknown whether pain duration has an influence on morphology of the pain and location, as well as whether morphology of pain and location have an influence on pain intensity.
The relation between pain maps and pain duration or pain intensity may be complex, because the perceived PFP is subjective, and considered as multifactorial \citep{Dansie2013}. Additionally, the study by \citeauthor{Boudreau2017} \citep{Boudreau2017} did not find a correlation between 35 pain maps and pain duration or pain intensity for individuals with a pain duration below 5 years. To investigate the potential nonlinear correlation, a deep learning method was used, which previously has not been applied on this type of data. \newline
\noindent
The goals of this study was to explore how accurate a deep learning model can classify pain maps according to pain duration or pain intensity. It was assumed that pain duration would be a better predictor than pain intensity, because of the subjectivity of pain, and its possibility of being multifactorial. 
The pain maps were encoded into multiple pain map representations to investigate whether morphology of the pain or location were correlated to pain duration or pain intensity.\newline
\noindent
It is assumed that a deep learning model will perform better with more features, thus a combined-representation containing morphology and location of the pain was made. The representations are referred to as morphology- (MR), location- (LR), and combined-representation (CR).
\noindent
The aim of this study was to explore classification performance of deep learning models, using PFP maps as input to classify according to either pain duration or pain intensity, and to compare performance accuracy between the three pain map representations (MR, LR, and CR).