Patellofemoral pain (PFP) syndrome is a painful musculoskeletal condition that is presented as pain behind or around the patella \citep{Maclachlan2017, Smith2015}. PFP syndrome affects 6-7\% of adolescents, of whom two thirds are highly physically active \citep{Rathleff2015}. Additionally the prevalence is more than twice as high for females than males \citep{Rathleff2015, Petersen2013}.
PFP syndrome may be present over a longer period of time where a high number of individuals experience a recurrent or chronic pain \citep{Witvrouw2014}. Chronic pain may be maintained by the phenomenon central sensitization, which may result widespread pain over longer periods of time. Furthermore, PFP syndrome may lead to osteoarthritis \citep{Petersen2013, Crossley2016}. 

\noindent
Patellofemoral pain (PFP) is often described as diffuse knee pain, that can be hard for individuals to explain and localize \citep{Witvrouw2014}. Despite the fact that individuals feel pain in the knee, there is no structural changes in the knee such as significant chondral damage. Because PFP is not caused by structural changes, no definitive clinical test may be used to diagnose PFP syndrome and thereby often diagnosed based on exclusion criterias \citep{Petersen2013} to which PFP syndrome is also described as an orthopaedic enigma, and is one of the most challenging pathologies to manage \citep{Dye2001}.
To assist diagnosis of PFP syndrome, pain maps may be used as a helpful tool for the individuals to communicate their pain by drawing pain areas on a body outline \citep{Boudreau2016}.

\noindent
A study by \citeauthor{Boudreau2017} \citep{Boudreau2017} indicates, through the use of pain maps, that there is a correlation between the size of the pain (number of pain pixels) and the pain duration as well as intensity for individuals with PFP longer than five years.\citep{Boudreau2017}
However, it is unknown whether pain duration has an influence on morphology of the pain and location, as well as  whether morphology of pain and location have an influence on pain intensity.
The relation between pain maps and pain duration or pain intensity may be complex, because the perceived PFP is subjective, and considered as multifactorial \citep{Dansie2013}. Additionally the study by \citeauthor{Boudreau2017} \citep{Boudreau2017} did not find a fully correlation between 35 pain maps and pain duration or pain intensity for individuals with a pain duration below 5 years. To investigate the potential nonlinear correlation, a deep learning method was used, which is a method that previously has not been applied on this type of data. \newline
\noindent
The goals of this study is to explore how accurate a deep learning model can classify pain maps according to pain duration or pain intensity. It is assumed that pain duration is a better predictor than pain intensity, because the perceived pain is subjective, and may be affected by multidimensional factors. 
The pain maps are encoded into multiple data representations to investigate whether morphology and location are correlated to pain duration or intensity.\newline
\noindent
The data representations are encoded into three representations, which reflect either morphology of pain or location. It is assumed that a deep learning model will perform better with more information, thus a combination of morphology and location of the pain constitute a data representation.  The data representations are refereed to as morphology-, location- and combined-representation.
There may be a difference in how gender reports pain intensity, where females reports more intense and frequent pain \citep{Pieh2012}. Furthermore, there is an imbalance in prevalence between females and males, thus gender is included as a feature in the deep learning model. \newline
\noindent
The aim of this study was to explore classification performance of a deep learning model, using PFP maps as input to classify according either pain duration or intensity.
Furthermore, a secondary aim was to compare the performance accuracy with different pain map representations (morphology-, location- and combined-representation), when predicting pain duration or pain intensity. 