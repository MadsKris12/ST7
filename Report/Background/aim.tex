Patellofemoral pain syndrome (PFPS) is a painful musculoskeletal condition that is presented as pain behind or around patella \citep{Maclachlan2017, Smith2015}. PFPS affects 6-7 \% of adolescents, of whom two thirds are highly physically active \citep{Rathleff2015}. Additionally the prevalence is more than twice as high for females than males.\citep{Petersen2013, Rathleff2015}.
PFPS may be present over longer periods of time where a high number of individuals experience a recurrent or chronic pain \citep{Witvrouw2014} and may also lead to osteoarthritis \citep{Petersen2013, Crossley2016}.

\noindent
Patellofemoral pain (PFP) is often described as diffuse knee pain, that can be hard to explain and localize \citep{Witvrouw2014}. Despite the fact that patients feel pain in the knee, there is not any structural changes in the knee such as significant chondral damage or increased Q-angle. There is no definitive clinical test to diagnose PFPS and it is thereby often diagnosed on exclusion criteria \citep{Petersen2013} to which PFPS is also described as an orthopaedic enigma, and is one of the most challenging pathologies to manage \citep{Dye2001}. 
To assist diagnosis of PFPS, pain maps may be used as a helpful tool for the individuals to communicate their pain by drawing pain areas \citep{Boudreau2016}. A study shows that through the use of pain maps it is possible to find a correlation between the symptom duration and the size and morphology of pain area \citep{Boudreau2017}. 
Another method to measure pain is by using visual analog scale (VAS), that scores pain between no-pain to the worst pain imaginable \citep{Haefeli2005}. However it is a known problem that chronic pain is considered as a multidimensional pain, because the perceived pain of an individual is influenced by biomedical, psychosocial and behavioral factors \citep{Dansie2013}

\noindent
Since PFP is associated with a lack of knowledge, and it has been shown that there is a correlation between pain maps and duration and pain intensity, it is interesting to investigate if pain maps can be used to classify and predict PFP related information. 

\noindent
A method that has not been found used in this context before is a deep learning. The deep learning method is chosen for this study because it is a state of the art method, that has shown greater performance in specific computation fields, compared to other machine learning methods \citep{LeCun2015}.
Furthermore the method is chosen because of its ability to find a non-linear connection between input and output data \citep{LeCun2015}, which is found relevant for this study mainly based on the fact that PFP is subjective and may be affected by the multidimensionality of chronic pain.  \\



\noindent
The goals of this project is to explore how accurate a deep learning model can classify symptom duration and pain intensity associated to PFP pain maps using a limited dataset. Because the prevalence is more than twice as high for females than males, the gender is included as an input parameter for the model. 
Furthermore morphology of the pain maps is considered to be relevant, based on the indication that morphology and size of pain area increase with prolonged symptom duration. 
To investigate the influence of morphology three types of pain map representations are created: a binary representation, a simplified representation based on knee regions and a combined representation that contains binary representations divided into knee regions.   

\section{Primary aim}
The aim of this study is to explore classification performance of a deep learning model, using PFP pain maps and gender as predictors to symptom duration and pain intensity. 
\vspace{4mm}
\noindent
\begin{center}
\textit{It is hypothesized that classification performance of the deep learning model is higher when using pain maps and gender to predict symptom duration than pain intensity.}
\end{center}

\section{Secondary aim}
The further aim of this project is to investigate if multiple pain map representations affect the deep learning model classification performance.

\vspace{4mm}
\noindent
\begin{center}
\textit{It is hypothesized that different data representations of pain maps affect the performance
accuracy of a deep learning model as related to the classification of symptom duration and pain intensity.
}
\end{center}









