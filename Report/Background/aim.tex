
As explained in the previous chapter, there are many possible causes for patellofemoral pain syndrome. Frederic (2012) states that this condition is associated with incorrect movement of the patella, that occurs when movement is outside of its ordinary track. <<<we have to add another source with possible cause>> But in the most cases, the etiology of PFPS is still unknown \citep{Smith2015}, what makes this disorder so interesting to investigate further. 

Patients are perceiving the pain which could be nociceptive, caused by external stimuli or neuropathic which remains after the damage of tissue is healed. In addition, to understand the possible causes of this pain, Parmer (1949) introduced a technique, which is used to transfer pain into the map by drawing it on a particular area of the knee. Unfortunately, there is not a lot of information to properly understand these maps. To be able to address this matter, a research question is created,
For this thesis, the research question will be answered through supporting hypotheses. Whether hypothesis will be valid or invalid, it will constitute to a general answer to the research question. The research question and the hypothesis will form a basis for this project.
\section{Research Statement}
The research statement for this thesis is as follows:\\
\noindent

\textit{Patellofemoral pain maps can be deeper investigated using neural network technique by classifying pain areas by the duration of the syndrome.}\\

The research aim of this thesis is to make an accurate classifier with a deep learning neural network using a limited amount of data. The information available with the pain maps will be pain maps features and used as output layer in the network. In this project, the duration of the pain was chosen in order to make a working classifier. The purpose is to make a reasonable study whether or not this technique could be used for classifying such kind of data. Thus, the intention is first to identify if it can be classify according to the set parameters and then secondly to see how accurate it can be applying different models, optimizers, and data preprocessing steps.
\section{Hypothesis}
The hypothesis is created to give a more comprehensive answer to the research question:\\
\noindent

\textit{It is hypothesized that deep neural network performs more accurate using superimposed images than raw data.}
  
%This makes it possible to use deep neural network for predicting the duration of the subjects pain (OOOOO REWRITE - WE DON'T KNOW IF WE CAN PREDICT OOOO). 
%In addition to this model a simple correlation model is used to compare the accuracy of the models. 
%Based on this the following hypothesis is formulated.\\

%\noindent
%Hypothesis: \textit{It is hypothesized that deep neural network performs more accurate than a simple correlation, when predicting the duration of patellofemoral pain from active pain regions.}
