Patellofemoral pain is a musculoskeletal condition that is presented as pain behind or around patella. Generally, pain is defined as an unpleasant sensory experience associated with potential or actual tissue damage, and it can be divided into different categories of pain depending on duration and location of the pain. Subjects with PFPS often describe the knee pain as being diffuse and hard to locate. PFPS is often classified as a chronic pain when the duration of the condition is longer than six months. Despite the feeling of pain in the knee, there is often no structural changes in the knee and therefore no definitive clinical test that can be used to diagnose PFPS. The diagnosis of PFPS is then often based on the perceived pain in the knee when doing pain provoking exercises. There are different methods for measuring pain, but since pain and pain intensity are very subjective and hence perceived differently, there are few reliable methods for objectively measuring pain. A method used to transfer a subject’s perceived pain into a relative objective illustration of the pain area is pain mapping, where subjects draw their pain areas on a body outline of the knee.\\

\noindent
The aim of this project is to test whether or not it is possible to classify variables associated to subjects with PFPS through the use of deep neural network, and a limited dataset. Furthermore different types of classifications will be tested to see how the variables affect the performance of the network.   
The choice of using a deep neural network solution is based on the fact that pain is subjective, from which it is assumed that there is no linear relation between the subjects. Furthermore it is not in the scope of this project to find the dominating features used for classification, to where the neural network is chosen since it's able to discover these features itself. Thereby is the aim of this project again only to see if a classification can be made and optimised. 
Classifications will be based on the gender, and how the subject perceive PFP expressed through pain maps. The morphology of the pain maps are considered to be a possible contributor for correct classification. To test this, the pain maps will be represented in different ways: pain morphology, pain regions and a combination of these.
According to \citeauthor{Boudreau2017} there is a correlation between the duration of PFPS above and below five years and the pain areas. SOMETHING ABOUT WHY WE WANT TO USE PAIN INTENSITY? SHELLIE WE NEED YOUR HELP!! 

\subsection{Hypothesis}
From the previous section the following hypothesis is given.

\noindent
\textit{It is hypothesized that different data representations of pain maps will affect the performance accuracy of a neural network, as well as the classification between either duration or pain intensity.}
  
%This makes it possible to use deep neural network for predicting the duration of the subjects pain (OOOOO REWRITE - WE DON'T KNOW IF WE CAN PREDICT OOOO). 
%In addition to this model a simple correlation model is used to compare the accuracy of the models. 
%Based on this the following hypothesis is formulated.\\

%\noindent
%Hypothesis: \textit{It is hypothesized that deep neural network performs more accurate than a simple correlation, when predicting the duration of patellofemoral pain from active pain regions.}
