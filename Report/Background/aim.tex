
In this project the focus is patellofemoral pain and how accurate the duration of the pain can be predicted by using different methods. PFP occurs as diffuse knee pain there makes it hard for patients to explain and point out the precise pain area. Furthermore it is hard for healthcare personnel to interpret and give a treatment for the conditions. Since the pain is perceived as diffuse and seems very different from person to person, is it assumed that the data is non-linear. To compensate for the differences in pain maps it is chosen to superimpose an atlas of the knee to the pain maps to help define the different pain regions with labels. This makes it possible to use deep neural network to predict the duration of the subjects pain. In addition to this model a simple correlation model is used to compare the accuracy of the models. Based on this the following hypothesis is formulated.\\

\noindent
Hypothesis: It is hypothesized that deep neural network performs more accurate than a simple correlation, when predicting the duration of patellofemoral pain from active pain regions.
