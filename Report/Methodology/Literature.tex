This chapter is created to give an overview of the worksheet’s structure and how the literature is found described. 
The purpose of this worksheet is to list the fundamental understanding needed and the findings from which the scientific article is formed. 

\section{Structure of the worksheet}
The worksheet is structured after background knowledge that contains anatomy of the knee, pain and pain maps, knee pain regions, machine learning and deep learning. The knowledge of the anatomic of the knee, pain and pain maps support the understanding of patellofemoral pain and the given pain maps. Machine learning and deep learning is necessary for development of the model to process the data. \\
\noindent
After the chapter Background is the aim of the project presented, which directs to the chapter Materials, where the data and program for developing the deep learning model is described. Next is the pre-processing elaborated followed by the data processing and results. 

\section{Literature}
For collecting literature is there used unstructured and structured searching. The unstructured is primary used in the beginning of the project, where the approach to the literature was wide and nonspecific. This was used to get a lot of keywords to further structured searching. The structured searching is used to develop the worksheet and the scientific article. Literature is primarily new and peer reviewed articles or textbooks.

