This chapter is an overview of the worksheet’s structure and will describe how literature is found. 
The purpose of the worksheet is to list the fundamental understanding and the findings from which the scientific article is formed. 

\section{Structure of the worksheet}
The worksheet is divided into different chapters: Background, Aim of the project, Materials, Data processing and results. The Background chapter contains anatomy of the knee, description of pain and pain maps, knee regions, machine learning and deep learning. The knowledge of the anatomy of the knee, pain and pain maps support the understanding of patellofemoral pain and the given pain maps. Machine learning and deep learning is necessary for development of the model to process the data. \\
\noindent
The aim of the project is presented and contains the formulated hypothesis. The description of the data and program for developing the deep learning model is written in the Materials chapter. The pre-processing is elaborated followed by the data processing and results. 

\section{Literature}

The literature search has been both unstructured and structured. The unstructured searching is primary used in the beginning of the project, where the approach to the literature was wide and nonspecific. This was used to get an overall understanding of the different aspects of the project and to gather keywords which later is used in the structured searching. The structured searching is used to develop the worksheet and the scientific article. Literature is primarily peer reviewed articles or textbooks.

