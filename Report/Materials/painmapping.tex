\section{Pain mapping}
Pain mapping is a technique that is used to transfer a patient’s perceived pain into an objective graph or map by drawing the pain area. Pain maps can be made by the patients who draw their pain areas on a display on which a body outline is shown, or it can be made by observers who observe the patients and then draw from the signs the patients are showing. Pain maps can consist of only the drawings, but sometimes a questionnaire is added to get a more detailed overview of the pain.\citep{Schott2010}


Pain mapping are commonly used in clinical practice \citep{Schott2010}, and can be useful for patients when they try to communicate their pain. Pain maps may also be helpful in diagnosing patients and follow-ups during or after treatment to get an indicator of the patient’s response to the treatment.\citep{Boudreau2016}
According to \citeauthor{Schott2010} there is some issues with the graphical representations of pain, some of which are problems with drawing a three-dimensional feeling of pain on a two-dimensional surface, and distinguishing between internal and external perceived pain on a map.\citep{Schott2010}
Lately there has been made some research about digital pain drawings where the drawings are done on a tablet or smartphone, either in 2-D like the paper versions or in 3-D body schemas.\citep{Boudreau2016}

Benefits of using digital pain drawings instead of paper drawings are according to the study by \citeauthor{Boudreau2016} the reduced human error found when evaluating the pain areas/pain maps. It is also timesaving as there is no need to transfer the paper drawings into digital records. The study concludes furthermore that the benefits of 3-D pain drawings are to be explored further.\citep{Boudreau2016}
There are several studies that examines pain patterns. Another study by \citeauthor{Boudreau2017} uses pain drawings obtained by using three-dimensional body schemas to locate pain areas in the knees to investigate pain patterns \citep{Boudreau2017}. Studies about pain patterns have been made for other parts of the body as well, for example the study by \citeauthor{Bayam2017} which considers pain patterns in the shoulder. In this study it is investigated whether pain mapping and pain patterns of shoulder pain can be useful in a clinical setting, and it is concluded that it would be useful in the primary and secondary sector and for research as well.\citep{Bayam2017}
